%!TEX TS-program = xelatex
%!TEX encoding = UTF-8 Unicode
% Awesome CV LaTeX Template for CV/Resume
%
% This template has been downloaded from:
% https://github.com/posquit0/Awesome-CV
%
% Author:
% Claud D. Park <posquit0.bj@gmail.com>
% http://www.posquit0.com
%
% Template license:
% CC BY-SA 4.0 (https://creativecommons.org/licenses/by-sa/4.0/)
%


%-------------------------------------------------------------------------------
% CONFIGURATIONS
%-------------------------------------------------------------------------------
% A4 paper size by default, use 'letterpaper' for US letter
\documentclass[11pt, a4paper]{awesome-cv}

% Configure page margins with geometry
\geometry{left=1.4cm, top=.8cm, right=1.4cm, bottom=1.8cm, footskip=.5cm}

% Specify the location of the included fonts
\fontdir[fonts/]

% Color for highlights
% Awesome Colors: awesome-emerald, awesome-skyblue, awesome-red, awesome-pink, awesome-orange
%                 awesome-nephritis, awesome-concrete, awesome-darknight
\colorlet{awesome}{awesome-red}
% Uncomment if you would like to specify your own color
% \definecolor{awesome}{HTML}{CA63A8}

% Colors for text
% Uncomment if you would like to specify your own color
% \definecolor{darktext}{HTML}{414141}
% \definecolor{text}{HTML}{333333}
% \definecolor{graytext}{HTML}{5D5D5D}
% \definecolor{lighttext}{HTML}{999999}

% Set false if you don't want to highlight section with awesome color
\setbool{acvSectionColorHighlight}{true}

% If you would like to change the social information separator from a pipe (|) to something else
\renewcommand{\acvHeaderSocialSep}{\quad\textbar\quad}

\newfontfamily\koreanfont[Path = /usr/share/fonts/truetype/nanum/]{NanumGothic}

%-------------------------------------------------------------------------------
%	PERSONAL INFORMATION
%	Comment any of the lines below if they are not required
%-------------------------------------------------------------------------------
% Available options: circle|rectangle,edge/noedge,left/right
\photo[rectangle]{./examples/profile.jpg}
\name{Jiun}{Jeong}
\position{M.S. in Computer Science}
\address{14, Pangyo-ro 319beon-gil, Bundang-gu, Seongnam-si, Gyeonggi-do, 13488, Republic of Korea}

\mobile{(+82) 010-4399-2048}
\email{jiun.jeong.cs@gmail.com}
\github{programmer-k}
%\homepage{www.posquit0.com}
%\linkedin{posquit0}
% \gitlab{gitlab-id}
% \stackoverflow{SO-id}{SO-name}
% \twitter{@twit}
% \skype{skype-id}
% \reddit{reddit-id}
% \medium{madium-id}
% \googlescholar{googlescholar-id}{name-to-display}
%% \firstname and \lastname will be used
\googlescholar{i14joPEAAAAJ}{}
% \extrainfo{extra informations}

\quote{I received a Master's degree in Computer Science at Yonsei University.
My research focused on concurrent data structures for persistent memory.
Specifically, I worked on the design and testing of the persistent, batch-based, lock-free queue for Intel Optane memory.
My research involved developing scalable and efficient non-blocking data structures, ensuring persistence guarantees,
devising safe memory reclamation schemes tailored for these data structures, and designing recovery mechanisms
to maintain data integrity and consistency in the event of system failures.}

%-------------------------------------------------------------------------------
%	BIBLIOGRAPHY
%-------------------------------------------------------------------------------
%\addbibresource{biblatex-examples.bib}
\addbibresource{cv/references.bib}

%-------------------------------------------------------------------------------
\begin{document}

{\koreanfont [전문연구요원 신규 편입 희망]}

% Print the header with above personal informations
% Give optional argument to change alignment(C: center, L: left, R: right)
\makecvheader

% Print the footer with 3 arguments(<left>, <center>, <right>)
% Leave any of these blank if they are not needed
\makecvfooter
  {Last Updated: \today}
  {Jiun Jeong~~~·~~~Curriculum Vitae}
  {\thepage}


%-------------------------------------------------------------------------------
%	CV/RESUME CONTENT
%	Each section is imported separately, open each file in turn to modify content
%-------------------------------------------------------------------------------
%-------------------------------------------------------------------------------
%	SECTION TITLE
%-------------------------------------------------------------------------------
\cvsection{Education}


%-------------------------------------------------------------------------------
%	CONTENT
%-------------------------------------------------------------------------------
\begin{cventries}

%---------------------------------------------------------
  \cventry
    {M.S. in \href{https://cs.yonsei.ac.kr/index.php}{Computer Science}} % Degree
    {\href{https://www.yonsei.ac.kr/sc/index.jsp}{Yonsei University}} % Institution
    {Republic of Korea} % Location
    {Mar. 2022 - Present} % Date(s)
    {
      \begin{cvitems} % Description(s) bullet points
        \item {Supervisor: \href{https://cs.yonsei.ac.kr/bbs/board.php?bo_table=sub2_1_a&wr_id=18}{Prof. Bernd Burgstaller}}
        \item {\href{https://elc.yonsei.ac.kr}{Embedded Systems Languages and Compilers (ELC) Lab}}
      \end{cvitems}
    }
%---------------------------------------------------------

%---------------------------------------------------------
  \cventry
    {B.S. in \href{https://cs.yonsei.ac.kr/index.php}{Computer Science}} % Degree
    {\href{https://www.yonsei.ac.kr/sc/index.jsp}{Yonsei University}} % Institution
    {Republic of Korea} % Location
    {Mar. 2017 - Feb. 2022} % Date(s)
    {
      \begin{cvitems} % Description(s) bullet points
        \item {Overall GPA: 3.53/4.3, Major GPA: 3.92/4.3}
        %\item {GPA: 4.13/4.3 for CSI courses}
      \end{cvitems}
    }
%---------------------------------------------------------
\end{cventries}

%-------------------------------------------------------------------------------
%	SECTION TITLE
%-------------------------------------------------------------------------------
\cvsection{Experience}


%-------------------------------------------------------------------------------
%	CONTENT
%-------------------------------------------------------------------------------
\begin{cventries}

%---------------------------------------------------------
  \cventry
    {Undergraduate Research Intern Supervised by Prof. Bernd Burgstaller} % Job title
    {ELC Lab at Yonsei University} % Organization
    {Seoul, South Korea} % Location
    {Jan. 2020 - Feb. 2022} % Date(s)
    {
      \begin{cvitems} % Description(s) of tasks/responsibilities
        \item {Profiling techniques for streaming frameworks in the cloud}
        \item {Heterogeneous memory architectures}
        \item {Testing and debugging support for Ethereum smart contracts at scale}
      \end{cvitems}
    }

%---------------------------------------------------------
\end{cventries}

%-------------------------------------------------------------------------------
%	SECTION TITLE
%-------------------------------------------------------------------------------
\cvsection{Publications}



% Will be done based on https://github.com/ravijo/Awesome-CV which supports BibTeX.

%%-------------------------------------------------------------------------------
%	SECTION TITLE
%-------------------------------------------------------------------------------
\cvsection{Projects}


%-------------------------------------------------------------------------------
%	CONTENT
%-------------------------------------------------------------------------------
\begin{cventries}

%---------------------------------------------------------
  \cventry
    {Samsung DS Division through the Yonsei-Samsung Semiconductor Research Center (YSSRC)} % Funding
    {Smart Near-far Memory Architecture for Data-intensive Workloads} % Title
    {Republic of Korea} % Location
    {Sep. 2020 - Present} % Date(s)
    {
      \begin{cvitems} % Description(s) of tasks/responsibilities
      \end{cvitems}
    }

  \cventry
    {Institute of Information \& Communications Technology Planning \& Evaluation (IITP)} % Funding
    {Application and Toolchain Support for Processing-in-memory (PIM)} % Title
    {Republic of Korea} % Location
    {Apr. 2021 - Dec. 2023} % Date(s)
    {
      \begin{cvitems} % Description(s) of tasks/responsibilities
      \end{cvitems}
    }

  % 이종 멀티코어 기반의 클라우드 상에서 프로그래머 생산성 및 퍼포먼스를 위한 엑사스케일 빅 데이터 분석 플랫폼
  \cventry
    {National Research Foundation of Korea (NRF)} % Funding
    {An Exa-scale Big Data Analysis Platform for Programmer Productivity and Performance on Clouds of Heterogeneous Multicore} % Title
    {Republic of Korea} % Location
    {Jan. 2020 - Oct. 2020} % Date(s)
    {
      \begin{cvitems} % Description(s) of tasks/responsibilities
      \end{cvitems}
    }

%---------------------------------------------------------
\end{cventries}

%-------------------------------------------------------------------------------
%	SECTION TITLE
%-------------------------------------------------------------------------------
\cvsection{Patents}


%-------------------------------------------------------------------------------
%	CONTENT
%-------------------------------------------------------------------------------
\begin{cventries}

%---------------------------------------------------------
  \cventry
    {Bernd Burgstaller, Hyunmo Sung, Seongho Jeong, Jay Hwan Lee, \underline{Jiun Jeong}, and Shinhyung Yang} % Inventors
    {Apparatus for Optimizing Code for Utilization of Process-In-Memory} % Title
    {Republic of Korea} % Location
    {Nov. 2023} % Date(s)
    {
      \begin{cvitems} % Description(s) of tasks/responsibilities
        \item {Korean Patent Application Number: 10-2023-0169862}
        \item {DOI: \href{https://doi.org/10.8080/1020230169862}{\texttt{https://doi.org/10.8080/1020230169862}}}
      \end{cvitems}
    }

%---------------------------------------------------------
\end{cventries}

%-------------------------------------------------------------------------------
%	SECTION TITLE
%-------------------------------------------------------------------------------
\cvsection{Awards}


%-------------------------------------------------------------------------------
%	CONTENT
%-------------------------------------------------------------------------------
\begin{cventries}

%---------------------------------------------------------
  \cventry
    {Department of Computer Science, Yonsei University} % Award
    {Graduate Student Research Assistant Scholarship} % Event
    {Republic of Korea} % Location
    {Spring 2022 - Fall 2023} % Date(s)
    {
      \begin{cvitems} % 4.13/4.3
        \item {Awarded based on a GPA of 4.13/4.3 in CSI undergraduate courses}
        \item {Full scholarship covering four semesters}
      \end{cvitems}
    }

%--------------------------------------------------------
  \cventry
    {Software Capstone Design} % Award
    {Excellence Award} % Event
    {Republic of Korea} % Location
    {Spring 2021} % Date(s)
    % Automatic Scoring and Feedback For Precision Education Software
    {
      \begin{cvitems}
        %Department of Computer Science, Yonsei University
        \item {Graduation team project for the Department of Computer Science at Yonsei University}
        \item {Title: \textit{Automatic Scoring and Feedback For Precision Education Software}}
      \end{cvitems}
    }

%---------------------------------------------------------
  \cventry
    {Software Capstone Design} % Award
    {Excellence Award} % Event
    {Republic of Korea} % Location
    {Fall 2020} % Date(s)
    % Statically Analyzing Access Vectors of Linux Kernel Vulnerabilities on Android Platforms
    {
      \begin{cvitems}
        %Department of Computer Science, Yonsei University
        \item {Graduation team project for the Department of Computer Science at Yonsei University}
        \item {Title: \textit{Statically Analyzing Access Vectors of Linux Kernel Vulnerabilities on Android Platforms}}
      \end{cvitems}
    }

%---------------------------------------------------------
\end{cventries}

%-------------------------------------------------------------------------------
%	SECTION TITLE
%-------------------------------------------------------------------------------
\cvsection{Teaching}


%-------------------------------------------------------------------------------
%	CONTENT
%-------------------------------------------------------------------------------
\begin{cventries}

%---------------------------------------------------------
  \cventry
    {Teaching Assistant} % Role
    {[CSI4104-01] Compiler Design} % Course Name
    {Seoul, South Korea} % Location
    {Fall 2023} % Date(s)
    {}

  \cventry
    {Teaching Assistant} % Role
    {[CCO1100-01] Computer Programming} % Course Name
    {Seoul, South Korea} % Location
    {Spring 2023} % Date(s)
    {}

  \cventry
    {Teaching Assistant} % Role
    {[CCO1100-02] Computer Programming} % Course Name
    {Seoul, South Korea} % Location
    {Spring 2023} % Date(s)
    {}

  \cventry
    {Teaching Assistant} % Role
    {[CCO1100-03] Computer Programming} % Course Name
    {Seoul, South Korea} % Location
    {Spring 2023} % Date(s)
    {}

  \cventry
    {Teaching Assistant} % Role
    {[CSI4104-01] Compiler Design} % Course Name
    {Seoul, South Korea} % Location
    {Fall 2022} % Date(s)
    {}

  \cventry
    {Teaching Assistant} % Role
    {[CAC1100-01] Computer Programming} % Course Name
    {Seoul, South Korea} % Location
    {Spring 2022} % Date(s)
    {}

  \cventry
    {Teaching Assistant} % Role
    {[CSI6505-01] Multicore Programming Fundamentals} % Course Name
    {Seoul, South Korea} % Location
    {Spring 2022} % Date(s)
    {}
%---------------------------------------------------------
\end{cventries}

%-------------------------------------------------------------------------------
%	SECTION TITLE
%-------------------------------------------------------------------------------
\cvsection{Language Proficiency}


%-------------------------------------------------------------------------------
%	CONTENT
%-------------------------------------------------------------------------------
\begin{cventries}

%---------------------------------------------------------
  \cventry
    {Certificate Expiration Date: 2027/02/16} % Status
    {TOEIC Speaking: \href{https://drive.google.com/file/d/1HgNQ4gUgvdH7-rlSQieMzz84lg-ij9-u/view?usp=sharing}{160 / 200, Advanced Low (AL)}} % Title
    {Republic of Korea} % Location
    {Feb. 2025} % Date(s)
    {
      %\begin{cvitems} % Description(s) of tasks/responsibilities
      %  \item {Pronunciation: high}
      %  \item {Intonation and stress: high}
      %\end{cvitems}
    }

  \cventry
    {Certificate Expiration Date: 2027/02/15} % Status
    {TOEIC: \href{https://drive.google.com/file/d/14gmf1MtT_eW6QlUEyug6A36HshiLP5mz/view?usp=sharing}{955 / 990}} % Title
    {Republic of Korea} % Location
    {Feb. 2025} % Date(s)
    {
      \begin{cvitems} % Description(s) of tasks/responsibilities
        \item {Listening Comprehension: 480 / 495}
        \item {Reading Comprehension: 475 / 495}
      \end{cvitems}
    }

  %  \item {OPIc: AL (Advanced Low), highest score for OPIc, test date 2024/09/05, certificate expiration date 2026/09/04 }
  \cventry
    {Certificate Expiration Date: 2026/09/04} % Status
    {OPIc: \href{https://drive.google.com/file/d/1WA33mSp0jaU5LKnqtJ-dRN5UOLOgsfLA/view?usp=sharing}{Advanced Low (AL)}} % Title
    {Republic of Korea} % Location
    {Sep. 2024} % Date(s)
    {
      \begin{cvitems} % Description(s) of tasks/responsibilities
        \item {Highest score for OPIc}
      \end{cvitems}
    }

  %\item {New TEPS: 472/600, Listening Comprehension: 189 / 240,Vocabulary: 40 / 60, Grammar: 42 / 60, Reading Comprehension: 201 / 240, TEST DATE 2021/09/04, VALID UNTIL 09 03 2023}
  \iffalse
  \cventry
    {Expired: No longer valid} % Status
    {New TEPS: 472 / 600} % Title
    {Republic of Korea} % Location
    {Sep. 2021} % Date(s)
    {
      \begin{cvitems} % Description(s) of tasks/responsibilities
        \item {Listening Comprehension: 189 / 240}
        \item {Vocabulary: 40 / 60}
        \item {Grammar: 42 / 60}
        \item {Reading Comprehension: 201 / 240}
      \end{cvitems}
    }

  % \item {TOEIC: 925/990, Listening Comprehension: 495/495, Reading Comprehension: 430/495, Test date: 2019/05/26, Valid until: 2021/05/26}
  \cventry
    {Expired: No longer valid} % Status
    {TOEIC: 925 / 990} % Title
    {Republic of Korea} % Location
    {May. 2019} % Date(s)
    {
      \begin{cvitems} % Description(s) of tasks/responsibilities
        \item {Listening Comprehension: 495 / 495}
        \item {Reading Comprehension: 430 / 495}
      \end{cvitems}
    }

  %\item {TOEFL: 104/120, Reading 27, Listening 28, Speaking 22, Writing 27, Test date 01 aug 2015, }
  \cventry
    {Expired: No longer valid} % Status
    {TOEFL iBT: 104 / 120} % Title
    {Republic of Korea} % Location
    {Aug. 2015} % Date(s)
    {
      \begin{cvitems} % Description(s) of tasks/responsibilities
        \item {Reading: 27 / 30}
        \item {Listening: 28 / 30}
        \item {Speaking: 22 / 30}
        \item {Writing: 27 / 30}
      \end{cvitems}
    }
  \fi

%---------------------------------------------------------
\end{cventries}


%-------------------------------------------------------------------------------
\end{document}
