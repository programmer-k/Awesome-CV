%!TEX TS-program = xelatex
%!TEX encoding = UTF-8 Unicode
% Awesome CV LaTeX Template for CV/Resume
%
% This template has been downloaded from:
% https://github.com/posquit0/Awesome-CV
%
% Author:
% Claud D. Park <posquit0.bj@gmail.com>
% http://www.posquit0.com
%
% Template license:
% CC BY-SA 4.0 (https://creativecommons.org/licenses/by-sa/4.0/)
%


%-------------------------------------------------------------------------------
% CONFIGURATIONS
%-------------------------------------------------------------------------------
% A4 paper size by default, use 'letterpaper' for US letter
\documentclass[11pt, a4paper]{awesome-cv}

% Configure page margins with geometry
\geometry{left=1.4cm, top=.8cm, right=1.4cm, bottom=1.8cm, footskip=.5cm}

% Specify the location of the included fonts
\fontdir[fonts/]

% Color for highlights
% Awesome Colors: awesome-emerald, awesome-skyblue, awesome-red, awesome-pink, awesome-orange
%                 awesome-nephritis, awesome-concrete, awesome-darknight
\colorlet{awesome}{awesome-red}
% Uncomment if you would like to specify your own color
% \definecolor{awesome}{HTML}{CA63A8}

% Colors for text
% Uncomment if you would like to specify your own color
% \definecolor{darktext}{HTML}{414141}
% \definecolor{text}{HTML}{333333}
% \definecolor{graytext}{HTML}{5D5D5D}
% \definecolor{lighttext}{HTML}{999999}

% Set false if you don't want to highlight section with awesome color
\setbool{acvSectionColorHighlight}{true}

% If you would like to change the social information separator from a pipe (|) to something else
\renewcommand{\acvHeaderSocialSep}{\quad\textbar\quad}


%-------------------------------------------------------------------------------
%	PERSONAL INFORMATION
%	Comment any of the lines below if they are not required
%-------------------------------------------------------------------------------
% Available options: circle|rectangle,edge/noedge,left/right
\photo[circle]{./examples/profile.png}
\name{Jiun}{Jeong}
\position{M.S. Candidate in Computer Science}
\address{D815, Engineering Hall 4, 50, Yonsei-ro, Seodaemun-gu, Seoul, 03722, Republic of Korea}

\mobile{(+82) 010-4399-2048}
\email{jiun.jeong@yonsei.ac.kr}
\github{programmer-k}
%\homepage{www.posquit0.com}
%\linkedin{posquit0}
% \gitlab{gitlab-id}
% \stackoverflow{SO-id}{SO-name}
% \twitter{@twit}
% \skype{skype-id}
% \reddit{reddit-id}
% \medium{madium-id}
% \googlescholar{googlescholar-id}{name-to-display}
%% \firstname and \lastname will be used
% \googlescholar{googlescholar-id}{}
% \extrainfo{extra informations}

\quote{``I am enrolled in the PhD/Master's combined degree program in Computer Science at Yonsei University. My research interests include concurrent data structures for persistent memory architectures, testing and debugging support for Ethereum smart contracts at scale, and profiling techniques for streaming frameworks in the cloud."}

%-------------------------------------------------------------------------------
%	BIBLIOGRAPHY
%-------------------------------------------------------------------------------
%\addbibresource{biblatex-examples.bib}
\addbibresource{cv/references.bib}

%-------------------------------------------------------------------------------
\begin{document}

% Print the header with above personal informations
% Give optional argument to change alignment(C: center, L: left, R: right)
\makecvheader

% Print the footer with 3 arguments(<left>, <center>, <right>)
% Leave any of these blank if they are not needed
\makecvfooter
  {Last Updated: \today}
  {Jiun Jeong~~~·~~~Curriculum Vitae}
  {\thepage}


%-------------------------------------------------------------------------------
%	CV/RESUME CONTENT
%	Each section is imported separately, open each file in turn to modify content
%-------------------------------------------------------------------------------
%-------------------------------------------------------------------------------
%	SECTION TITLE
%-------------------------------------------------------------------------------
\cvsection{Education}


%-------------------------------------------------------------------------------
%	CONTENT
%-------------------------------------------------------------------------------
\begin{cventries}

%---------------------------------------------------------
  \cventry
    {B.S. in Computer Science} % Degree
    {Yonsei University} % Institution
    {Seoul, South Korea} % Location
    {Mar. 2017 - Feb. 2022} % Date(s)
    {
      \begin{cvitems} % Description(s) bullet points
        \item {GPA: x.xx/4.3 for CSI courses}
      \end{cvitems}
    }

%---------------------------------------------------------
\end{cventries}

%-------------------------------------------------------------------------------
%	SECTION TITLE
%-------------------------------------------------------------------------------
\cvsection{Experience}


%-------------------------------------------------------------------------------
%	CONTENT
%-------------------------------------------------------------------------------
\begin{cventries}

%---------------------------------------------------------
  \cventry
    {AI SDK Software Engineer (Internship)} % Job title
    {\href{https://www.gaudiolab.com}{Gaudio Lab, Inc.}} % Organization
    {Republic of Korea} % Location
    {Feb. 2025 - Apr. 2025} % Date(s): 17 Feb - 7 Apr
    {
      \begin{cvitems} % Description(s) of tasks/responsibilities
        \item {Wrote unit and regression tests to validate SDK APIs}
        %\item {Authored developer-facing API documentation with usage examples in Markdown}
        %\item {Developed and tested WebAssembly-based browser integration for a C-based audio SDK}
        \item {Built the C-based audio SDK into WebAssembly and integrated it with AudioWorklet for browser-based audio processing}
        %\item {Measured real-time factor (RTF) and evaluated SIMD optimization effects in a demo webpage}
      \end{cvitems}
    }

%---------------------------------------------------------

%---------------------------------------------------------
  \cventry
    {Undergraduate Research Intern Supervised by \href{https://cs.yonsei.ac.kr/bbs/board.php?bo_table=sub2_1_a&wr_id=18}{Prof. Bernd Burgstaller}} % Job title
    {\href{https://elc.yonsei.ac.kr}{ELC Lab} at \href{https://www.yonsei.ac.kr/sc/index.jsp}{Yonsei University}} % Organization
    {Republic of Korea} % Location
    {Jan. 2020 - Feb. 2022} % Date(s)
    {
      \begin{cvitems} % Description(s) of tasks/responsibilities
        \item {Profiling techniques for streaming frameworks in the cloud}
        \item {Heterogeneous memory architectures}
        \item {Testing and debugging support for Ethereum smart contracts at scale}
      \end{cvitems}
    }

%---------------------------------------------------------
\end{cventries}

%-------------------------------------------------------------------------------
% SECTION TITLE
%-------------------------------------------------------------------------------
\cvsection{Publications}

%-------------------------------------------------------------------------------
% SUBSECTION TITLE - International Journals
%-------------------------------------------------------------------------------
\cvsubsection{International Journals}

\begin{refsection}
    \nocite{10876606}

    \printbibliography[
    heading=none, 
    sorting=ydnt
    ]
\end{refsection}

%-------------------------------------------------------------------------------
% SUBSECTION TITLE - Conference Proceedings
%-------------------------------------------------------------------------------
%\cvsubsection{Conference Proceedings}
%
%\begin{refsection}
%    
%    \printbibliography[
%    heading=none, 
%    sorting=ydnt
%    ]
%\end{refsection}

%-------------------------------------------------------------------------------
% SUBSECTION TITLE - Preprints (arXiv)
%-------------------------------------------------------------------------------
\cvsubsection{Preprints}

\begin{refsection}
    \nocite{yang2023cloudprofiler}

    \printbibliography[
    heading=none, 
    sorting=ydnt
    ]
\end{refsection}

%-------------------------------------------------------------------------------
% SUBSECTION TITLE - Domestic Conferences
%-------------------------------------------------------------------------------
\cvsubsection{Domestic Conferences}

\begin{refsection}
    \nocite{jeong2024workstation}
    \printbibliography[
    heading=none, 
    sorting=ydnt
    ]
\end{refsection}

%-------------------------------------------------------------------------------
%	SECTION TITLE
%-------------------------------------------------------------------------------
\cvsection{Awards}



%-------------------------------------------------------------------------------
%	CONTENT
%-------------------------------------------------------------------------------
\begin{cvhonors}

%---------------------------------------------------------
  \cvhonor
    {Full Scholarship} % Award
    {Department of Computer Science, Yonsei University} % Event
    {Seoul, South Korea} % Location
    {2022} % Date(s)

%--------------------------------------------------------
  \cvhonor
    {Excellence Award} % Award
    {Software Capstone Design, Department of Computer Science, Yonsei University} % Event
    {Seoul, South Korea} % Location
    {2021} % Date(s)

%---------------------------------------------------------
  \cvhonor
    {Excellence Award} % Award
    {Software Capstone Design, Department of Computer Science, Yonsei University} % Event
    {Seoul, South Korea} % Location
    {2020} % Date(s)

%---------------------------------------------------------
\end{cvhonors}

%-------------------------------------------------------------------------------
%	SECTION TITLE
%-------------------------------------------------------------------------------
\cvsection{Teaching}


%-------------------------------------------------------------------------------
%	CONTENT
%-------------------------------------------------------------------------------
\begin{cventries}

%---------------------------------------------------------
  \cventry
    {Teaching Assistant} % Role
    {[CCO1100-01] Computer Programming} % Course Name
    {Seoul, South Korea} % Location
    {Spring 2023} % Date(s)
    {}

  \cventry
    {Teaching Assistant} % Role
    {[CCO1100-02] Computer Programming} % Course Name
    {Seoul, South Korea} % Location
    {Spring 2023} % Date(s)
    {}

  \cventry
    {Teaching Assistant} % Role
    {[CCO1100-03] Computer Programming} % Course Name
    {Seoul, South Korea} % Location
    {Spring 2023} % Date(s)
    {}

  \cventry
    {Teaching Assistant} % Role
    {[CSI4104-01] Compiler Design} % Course Name
    {Seoul, South Korea} % Location
    {Fall 2022} % Date(s)
    {}

  \cventry
    {Teaching Assistant} % Role
    {[CAC1100-01] Computer Programming} % Course Name
    {Seoul, South Korea} % Location
    {Spring 2022} % Date(s)
    {}

  \cventry
    {Teaching Assistant} % Role
    {[CSI6505-01] Multicore Programming Fundamentals} % Course Name
    {Seoul, South Korea} % Location
    {Spring 2022} % Date(s)
    {}
%---------------------------------------------------------
\end{cventries}

\begin{cvitems}
  \item {TOEFL: TODO/120}
  \item {TOEIC: 925/990, Listening Comprehension: 495/495, Reading Comprehension: 430/495, Test date: 2019/05/26, Valid until: 2021/05/26}
  \item {New TEPS: 472/600, Listening Comprehension: 189 / 240,Vocabulary: 40 / 60, Grammar: 42 / 60, Reading Comprehension: 201 / 240, TEST DATE 2021/09/04, VALID UNTIL 09 03 2023}
  \item {OPIc: AL}
\end{cvitems}

%-------------------------------------------------------------------------------
%	SECTION TITLE
%-------------------------------------------------------------------------------
\cvsection{Patents}


%-------------------------------------------------------------------------------
%	CONTENT
%-------------------------------------------------------------------------------
\begin{cventries}

%---------------------------------------------------------
  \cventry
    {Bernd Burgstaller, Hyunmo Sung, Seongho Jeong, Jay Hwan Lee, \underline{Jiun Jeong}, and Shinhyung Yang} % Inventors
    {Apparatus for Optimizing Code for Utilization of Process-In-Memory} % Title
    {Republic of Korea} % Location
    {Nov. 2023} % Date(s)
    {
      \begin{cvitems} % Description(s) of tasks/responsibilities
        \item {Korean Patent Application Number: 10-2023-0169862}
        \item {DOI: \href{https://doi.org/10.8080/1020230169862}{\texttt{10.8080/1020230169862}}}
      \end{cvitems}
    }

%---------------------------------------------------------
\end{cventries}


% 이게 최신인거 같음, 이거만 검색됨.
%https://doi.org/10.8080/1020230169862
% 아니 2022년인거임 23년인거임;; 헷갈리네
%https://ysis2.yonsei.ac.kr/
% 어떻게 작성해야 할지 생각해보기. 신형님?
%프로세스 인 메모리의 활용을 위한 코드 최적화 방법 및 그를 위한 장치
% 출원번호:10-2023-0169862
% 프로세스 인 메모리의 활용을 위한 코드 최적화 방법 및 그를 위한 장치{Method and
%Apparatus for optimizing Code for Utilization of Process-In-Memory}
% https://sites.google.com/view/yonsei-medisyslab/publications/patent?authuser=0
% 이것도 참조


%프로세스 인 메모리의 활용을 위한 오프로드 처리 방법 및 그를 위한 장치
% 출원번호: 10-2022-0162906
% 번호로도 검색안됨

%-------------------------------------------------------------------------------
\end{document}
